%
% File:    Macros for Journal of Machine Learning Research
%      Very minor modification of macros for Journal of Artificial 
%      Intelligence Research (jair.sty) 
%
% Suggestions: Submit an issue or pull request to 
%   https://github.com/JournalMLR/jmlr-style-file
%
%   Last edited October 9, 2000 by Leslie Pack Kaelbling
%   Last edited January 23, 2001 by Alex J. Smola (we should set up RCS or CVS)
%   Last edited March 29, 2004 Erik G. Learned-Miller
%   Last edited January 17, 2016 Charles Sutton 
%   Last edited January 9, 2017 Charles Sutton 
%    (We have now set up GIT, good thing that we waited for it to
%     be invented.)
%
%          The name of this file should follow the article document
%          type, e.g. \documentstyle[jmlr]{article}

%          Copied and edited from similar file for Machine Learning Journal.
% Original Author:  Jeff Schlimmer
% Edited by: Kevin Thompson, Martha Del Alto, Helen Stewart, Steve Minton \& Pandu Nayak.
% Last edited: Mon  May  3 20:40:00 1993 by kthompso (Kevin Thompson) on muir

\typeout{Document Style `jmlr' -- January 2016.}

\newif\if@abbrvbib\@abbrvbibfalse
\DeclareOption{abbrvbib}{\@abbrvbibtrue}

\newif\if@usehyper\@usehypertrue
\DeclareOption{nohyperref}{\@usehyperfalse}
\DeclareOption{hyperref}{\@usehypertrue}

\DeclareOption*{\PackageWarning{jmlr}{Unknown ‘\CurrentOption’}}
\ProcessOptions\relax

%%%%%%%%%%%%%%%%%%%%%%%%%%%%%%%%%%%%%%%%%%%%%%%%%%%%%%%%%%%%%%%%%%%%%%%%%%%
%                             REQUIRED PACKAGES
%%%%%%%%%%%%%%%%%%%%%%%%%%%%%%%%%%%%%%%%%%%%%%%%%%%%%%%%%%%%%%%%%%%%%%%%%%%

\RequirePackage{epsfig}
\RequirePackage{amssymb}
\RequirePackage{natbib}
\RequirePackage{graphicx}

\if@usehyper
\RequirePackage[colorlinks=false,allbordercolors={1 1 1}]{hyperref}
\fi

\if@abbrvbib
\bibliographystyle{abbrvnat}
\else
\bibliographystyle{plainnat}
\fi

\bibpunct{(}{)}{;}{a}{,}{,}
 
%%%%%%%%%%%%%%%%%%%%%%%%%%%%%%%%%%%%%%%%%%%%%%%%%%%%%%%%%%%%%%%%%%%%%%%%%%%
%                             P A G E   S I Z E
%%%%%%%%%%%%%%%%%%%%%%%%%%%%%%%%%%%%%%%%%%%%%%%%%%%%%%%%%%%%%%%%%%%%%%%%%%%
 
% Change the overall width of the page.  If these parameters are
%       changed, they will require corresponding changes in the
%       maketitle section.
%
\renewcommand{\topfraction}{0.95}   % let figure take up nearly whole page
\renewcommand{\textfraction}{0.05}  % let figure take up nearly whole page

% Specify the dimensions of each page

\oddsidemargin .25in    %   Note \oddsidemargin = \evensidemargin
\evensidemargin .25in
\marginparwidth 0.07 true in
%\marginparwidth 0.75 true in
%\topmargin 0 true pt           % Nominal distance from top of page to top of
%\topmargin 0.125in
\topmargin -0.5in
\addtolength{\headsep}{0.25in}
\textheight 8.5 true in       % Height of text (including footnotes & figures)
\textwidth 6.0 true in        % Width of text line.
\widowpenalty=10000
\clubpenalty=10000
\@twosidetrue \@mparswitchtrue \def\ds@draft{\overfullrule 5pt}

%%%%%%%%%%%%%%%%%%%%%%%%%%%%%%%%%%%%%%%%%%%%%%%%%%%%%%%%%%%%%%%%%%%%%%%%%%%
%                               S E C T I O N S
%%%%%%%%%%%%%%%%%%%%%%%%%%%%%%%%%%%%%%%%%%%%%%%%%%%%%%%%%%%%%%%%%%%%%%%%%%%

% Definitions for nicer (?) sections, etc., ideas from Pat Langley.
% Numbering for sections, etc. is taken care of automatically.

\def\@startsiction#1#2#3#4#5#6{\if@noskipsec \leavevmode \fi
   \par \@tempskipa #4\relax
   \@afterindenttrue
   \ifdim \@tempskipa <\z@ \@tempskipa -\@tempskipa \@afterindentfalse\fi
   \if@nobreak \everypar{}\else
     \addpenalty{\@secpenalty}\addvspace{\@tempskipa}\fi \@ifstar
     {\@ssect{#3}{#4}{#5}{#6}}{\@dblarg{\@sict{#1}{#2}{#3}{#4}{#5}{#6}}}}

\def\@sict#1#2#3#4#5#6[#7]#8{\ifnum #2>\c@secnumdepth
     \def\@svsec{}\else
     \refstepcounter{#1}\edef\@svsec{\csname the#1\endcsname}\fi
     \@tempskipa #5\relax
      \ifdim \@tempskipa>\z@
        \begingroup #6\relax
          \@hangfrom{\hskip #3\relax\@svsec.\hskip 0.1em}
                    {\interlinepenalty \@M #8\par}
        \endgroup
       \csname #1mark\endcsname{#7}\addcontentsline
         {toc}{#1}{\ifnum #2>\c@secnumdepth \else
                      \protect\numberline{\csname the#1\endcsname}\fi
                    #7}\else
        \def\@svsechd{#6\hskip #3\@svsec #8\csname #1mark\endcsname
                      {#7}\addcontentsline
                           {toc}{#1}{\ifnum #2>\c@secnumdepth \else
                             \protect\numberline{\csname the#1\endcsname}\fi
                       #7}}\fi
     \@xsect{#5}}

\def\@sect#1#2#3#4#5#6[#7]#8{\ifnum #2>\c@secnumdepth
     \def\@svsec{}\else 
     \refstepcounter{#1}\edef\@svsec{\csname the#1\endcsname\hskip 0.5em }\fi
     \@tempskipa #5\relax
      \ifdim \@tempskipa>\z@ 
        \begingroup #6\relax
          \@hangfrom{\hskip #3\relax\@svsec}{\interlinepenalty \@M #8\par}
        \endgroup
       \csname #1mark\endcsname{#7}\addcontentsline
         {toc}{#1}{\ifnum #2>\c@secnumdepth \else
                      \protect\numberline{\csname the#1\endcsname}\fi
                    #7}\else
        \def\@svsechd{#6\hskip #3\@svsec #8\csname #1mark\endcsname
                      {#7}\addcontentsline
                           {toc}{#1}{\ifnum #2>\c@secnumdepth \else
                             \protect\numberline{\csname the#1\endcsname}\fi
                       #7}}\fi
     \@xsect{#5}}

\def\thesection {\arabic{section}}
\def\thesubsection {\thesection.\arabic{subsection}}
\def\section{\@startsiction{section}{1}{\z@}{-0.24in}{0.10in}
             {\large\bf\raggedright}}
\def\subsection{\@startsection{subsection}{2}{\z@}{-0.20in}{0.08in}
                {\normalsize\bf\raggedright}}
\def\subsubsection{\@startsection{subsubsection}{3}{\z@}{-0.18in}{0.08in}
                {\normalsize\sc\raggedright}}
\def\paragraph{\@startsiction{paragraph}{4}{\z@}{1.5ex plus
  0.5ex minus .2ex}{-1em}{\normalsize\bf}}
\def\subparagraph{\@startsiction{subparagraph}{5}{\z@}{1.5ex plus
  0.5ex minus .2ex}{-1em}{\normalsize\bf}}


%%%%%%%%%%%%%%%%%%%%%%%%%%%%%%%%%%%%%%%%%%%%%%%%%%%%%%%%%%%%%%%%%%%%%%%%%%%
%                              F O O T N O T E S
%%%%%%%%%%%%%%%%%%%%%%%%%%%%%%%%%%%%%%%%%%%%%%%%%%%%%%%%%%%%%%%%%%%%%%%%%%%

% Change the size of the footnote rule
%
% \renewcommand{\footnoterule}{\vspace{10pt}\hrule width 0mm}

\long\def\@makefntext#1{\@setpar{\@@par\@tempdima \hsize 
             \advance\@tempdima-15pt\parshape \@ne 15pt \@tempdima}\par
             \parindent 2em\noindent \hbox to \z@{\hss{\@thefnmark}. \hfil}#1}

%%%%%%%%%%%%%%%%%%%%%%%%%%%%%%%%%%%%%%%%%%%%%%%%%%%%%%%%%%%%%%%%%%%%%%%%%%%
%                               A B S T R A C T
%%%%%%%%%%%%%%%%%%%%%%%%%%%%%%%%%%%%%%%%%%%%%%%%%%%%%%%%%%%%%%%%%%%%%%%%%%%

%% use \begin{abstract} .. \end{abstract} for abstracts.
\renewenvironment{abstract}
{\centerline{\large\bf Abstract}\vspace{0.7ex}%
  \bgroup\leftskip 20pt\rightskip 20pt\small\noindent\ignorespaces}%
{\par\egroup\vskip 0.25ex}

%%%%%%%%%%%%%%%%%%%%%%%%%%%%%%%%%%%%%%%%%%%%%%%%%%%%%%%%%%%%%%%%%%%%%%%%%%%
%                               KEYWORDS
%%%%%%%%%%%%%%%%%%%%%%%%%%%%%%%%%%%%%%%%%%%%%%%%%%%%%%%%%%%%%%%%%%%%%%%%%%%

%% use \begin{keywords} .. \end{keywords} for keywordss.
\newenvironment{keywords}
{\bgroup\leftskip 20pt\rightskip 20pt \small\noindent{\bf Keywords:} }%
{\par\egroup\vskip 0.25ex}

%%%%%%%%%%%%%%%%%%%%%%%%%%%%%%%%%%%%%%%%%%%%%%%%%%%%%%%%%%%%%%%%%%%%%%%%%%%
%                        FIRST PAGE, TITLE, AUTHOR
%%%%%%%%%%%%%%%%%%%%%%%%%%%%%%%%%%%%%%%%%%%%%%%%%%%%%%%%%%%%%%%%%%%%%%%%%%%

% Author information can be set in various styles:
% For several authors from the same institution:
% \author{Author 1 \and ... \and Author n \\
%         \addr{Address line} \\ ... \\ \addr{Address line}}
% if the names do not fit well on one line use
%         Author 1 \\ {\bf Author 2} \\ ... \\ {\bf Author n} \\
% To start a seperate ``row'' of authors use \AND, as in
% \author{Author 1 \\ \addr{Address line} \\  ... \\ \addr{Address line}
%         \AND
%         Author 2 \\ \addr{Address line} \\ ... \\ \addr{Address line} \And
%         Author 3 \\ \addr{Address line} \\ ... \\ \addr{Address line}}

% Title stuff, borrowed in part from aaai92.sty

\newlength\aftertitskip     \newlength\beforetitskip
\newlength\interauthorskip  \newlength\aftermaketitskip

%% Changeable parameters.
\setlength\aftertitskip{0.1in plus 0.2in minus 0.2in}
\setlength\beforetitskip{0.05in plus 0.08in minus 0.08in}
\setlength\interauthorskip{0.08in plus 0.1in minus 0.1in}
\setlength\aftermaketitskip{0.3in plus 0.1in minus 0.1in}

%% overall definition of maketitle, @maketitle does the real work
\def\maketitle{\par
 \begingroup
   \def\thefootnote{\fnsymbol{footnote}}
   \def\@makefnmark{\hbox to 0pt{$^{\@thefnmark}$\hss}}
   \@maketitle \@thanks
 \endgroup
\setcounter{footnote}{0}
 \let\maketitle\relax \let\@maketitle\relax
 \gdef\@thanks{}\gdef\@author{}\gdef\@title{}\let\thanks\relax}

\def\@startauthor{\noindent \normalsize\bf}
\def\@endauthor{}
\def\@starteditor{\noindent \small {\bf Editor:~}}
\def\@endeditor{\normalsize}
\def\@maketitle{\vbox{\hsize\textwidth
 \linewidth\hsize \vskip \beforetitskip
 {\begin{center} \Large\bf \@title \par \end{center}} \vskip \aftertitskip
 {\def\and{\unskip\enspace{\rm and}\enspace}%
  \def\addr{\small\it}%
  \def\email{\hfill\small\sc}%
  \def\name{\normalsize\bf}%
  \def\AND{\@endauthor\rm\hss \vskip \interauthorskip \@startauthor}
  \@startauthor \@author \@endauthor}

  \vskip \aftermaketitskip
  \noindent \@starteditor \@editor \@endeditor
  \vskip \aftermaketitskip
}}

\def\editor#1{\gdef\@editor{#1}}

%%%%%%%%%%%%%%%%%%%%%%%%%%%%%%%%%%%%%%%%%%%%%%%%%%%%%%%%%%%%%%%%%%%%%%%%%%%
%%%
%%% Pagestyle
%%
%%%%%%%%%%%%%%%%%%%%%%%%%%%%%%%%%%%%%%%%%%%%%%%%%%%%%%%%%%%%%%%%%%%%%%%%%%%

%% Defines the pagestyle for the title page.
%% Usage: \jmlrheading{1}{1993}{1-15}{8/93}{9/93}{14-115}{Jane Q. Public and A. U. Thor}
%%        \jmlrheading{vol}{year}{pages}{Submitted date}{published date}{paper id}{authors}
%%
%% If your paper required revisions that were reviewed by the action editor, then indicate
%%  this by, e.g.
%%   \jmlrheading{1}{1993}{1-15}{8/93; Revised 10/93}{12/93}{14-115}{Jane Q. Public and A. U. Thor}

\def\firstpageno#1{\setcounter{page}{#1}}

\def\jmlrheading#1#2#3#4#5#6#7{\def\ps@jmlrtps{\let\@mkboth\@gobbletwo%
\def\@oddhead{\scriptsize Journal of Machine Learning Research #1 (#2) #3 \hfill Submitted #4; Published #5}%
\def\@oddfoot{\parbox[t]{\textwidth}{\raggedright \scriptsize \copyright #2 #7.\\[5pt] 
License: CC-BY 4.0, see \url{https://creativecommons.org/licenses/by/4.0/}. Attribution requirements
are provided at \url{http://jmlr.org/papers/v#1/#6.html}.\hfill}}%

\def\@evenhead{}\def\@evenfoot{}}%
\thispagestyle{jmlrtps}}

%% Defines the pagestyle for the rest of the pages
%% Usage: \ShortHeadings{Minimizing Conflicts}{Minton et al}
%%        \ShortHeadings{short title}{short authors}

\def\ShortHeadings#1#2{\def\ps@jmlrps{\let\@mkboth\@gobbletwo%
\def\@oddhead{\hfill {\small\sc #1} \hfill}%
\def\@oddfoot{\hfill \small\rm \thepage \hfill}%
\def\@evenhead{\hfill {\small\sc #2} \hfill}%
\def\@evenfoot{\hfill \small\rm \thepage \hfill}}%
\pagestyle{jmlrps}}


%%%%%%%%%%%%%%%%%%%%%%%%%%%%%%%%%%%%%%%%%%%%%%%%%%%%%%%%%%%%%%%%%%%%%%%%%%
%                               MISCELLANY
%%%%%%%%%%%%%%%%%%%%%%%%%%%%%%%%%%%%%%%%%%%%%%%%%%%%%%%%%%%%%%%%%%%%%%%%%%%

% Define macros for figure captions and table titles

% Figurecaption prints the caption title flush left.
% \def\figurecaption#1#2{\noindent\hangindent 42pt
%                        \hbox to 36pt {\sl #1 \hfil}
%                        \ignorespaces #2}
% \def\figurecaption#1#2{\noindent\hangindent 46pt
%                        \hbox to 41pt {\small\sl #1 \hfil}
%                        \ignorespaces {\small #2}}
\def\figurecaption#1#2{\noindent\hangindent 40pt
                       \hbox to 36pt {\small\sl #1 \hfil}
                       \ignorespaces {\small #2}}
% Figurecenter prints the caption title centered.
\def\figurecenter#1#2{\centerline{{\sl #1} #2}}
\def\figurecenter#1#2{\centerline{{\small\sl #1} {\small #2}}}

%
%  Allow ``hanging indents'' in long captions
%
\long\def\@makecaption#1#2{
   \vskip 10pt 
   \setbox\@tempboxa\hbox{#1: #2}
   \ifdim \wd\@tempboxa >\hsize               % IF longer than one line:
       \begin{list}{#1:}{
       \settowidth{\labelwidth}{#1:}
       \setlength{\leftmargin}{\labelwidth}
       \addtolength{\leftmargin}{\labelsep}
        }\item #2 \end{list}\par   % Output in quote mode
     \else                                    %   ELSE  center.
       \hbox to\hsize{\hfil\box\@tempboxa\hfil}  
   \fi}


% Define strut macros for skipping spaces above and below text in a
% tabular environment.
\def\abovestrut#1{\rule[0in]{0in}{#1}\ignorespaces}
\def\belowstrut#1{\rule[-#1]{0in}{#1}\ignorespaces}

% Acknowledgments
\long\def\acks#1{\vskip 0.3in\noindent{\large\bf Acknowledgments}\vskip 0.2in
\noindent #1}

% Research Note
\long\def\researchnote#1{\noindent {\LARGE\it Research Note} #1}

\renewcommand{\appendix}{\par
  \setcounter{section}{0}
  \setcounter{subsection}{0}
  \def\thesection{\Alph{section}}
\def\section{\@ifnextchar*{\@startsiction{section}{1}{\z@}{-0.24in}{0.10in}%
             {\large\bf\raggedright}}%
{\@startsiction{section}{1}{\z@}{-0.24in}{0.10in}
             {\large\bf\raggedright Appendix\ }}}}

%%%%%%%%%%%%%%%%%%%%%%%%%%%%%%%%%%%%%%%%%%%%%%%%%%%%%%%%%%%%%%%%%%%%%%%%%%
%                               PROOF, THEOREM, and FRIENDS
%%%%%%%%%%%%%%%%%%%%%%%%%%%%%%%%%%%%%%%%%%%%%%%%%%%%%%%%%%%%%%%%%%%%%%%%%%%

\newcommand{\BlackBox}{\rule{1.5ex}{1.5ex}}  % end of proof
\newenvironment{proof}{\par\noindent{\bf Proof\ }}{\hfill\BlackBox\\[2mm]}
\newtheorem{example}{Example} 
\newtheorem{theorem}{Theorem}
\newtheorem{lemma}[theorem]{Lemma} 
\newtheorem{proposition}[theorem]{Proposition} 
\newtheorem{remark}[theorem]{Remark}
\newtheorem{corollary}[theorem]{Corollary}
\newtheorem{definition}[theorem]{Definition}
\newtheorem{conjecture}[theorem]{Conjecture}
\newtheorem{axiom}[theorem]{Axiom}


